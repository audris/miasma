\documentstyle{article}
\setlength{\oddsidemargin}{0in}
\setlength{\evensidemargin}{0in}
\setlength{\textwidth}{6.33in}
\setlength{\topmargin}{0in}
\setlength{\headsep}{0in}
\setlength{\textheight}{8.5in}

\begin{document}
\section*{MALARIA}
             
 One of the earliest recorded human diseases, MALARIA is a
widespread and often fatal PROTOZOAL DISEASE.  It
occurs mostly in tropical regions of the world, such as
Africa, Central America, and Southeast Asia.  Experts
estimate there are more than 270 million new MALARIA
infections worldwide each year and an estimated 2
million deaths.

MALARIA means "bad air" in Italian, reflecting the pre-1880 
view that it is caused by gases from the swampy regions where 
many cases occur. In 1880, Charles LAVERAN observed a 
protozoal parasite in the blood of an afflicted patient. In 
1898, Ronald Ross found that the bite of the female MOSQUITO 
of the genus Anopheles transmits the parasites into the 
victim's bloodstream. 

Life Cycle

The parasites, once in the bloodstream, travel to the liver 
and reproduce asexually in cells in the liver.  These cells 
burst and release the parasites back into the bloodstream.  
The parasites then enter red blood cells, where they reproduce 
asexually, and feed on the hemoglobin in the cells.  The red 
blood cells burst, releasing both asexual and potential sexual 
forms of the parasite.  The sexual forms develop in the gut of 
Anopheles mosquitoes that have fed on infected blood, thus 
completing the parasite's life cycle.  This cycle causes the 
symptoms of intermittent fever and chills in MALARIA victims. 
Four types of PLASMODIUM parasites--belonging to the subphylum 
SPOROZOA--cause MALARIA, and the cycle of each type is 
completed on a different time period.  As a result, symptoms 
occur about every 24 hours when a victim is infected with P. 
falciparum, at 48-hour intervals with P. vivax and P. ovale, 
and at 72-hour intervals with P. malariae.  Other malarial 
symptoms include headache, weakness, and an enlarged spleen 
(see SPLENOMEGALY). 

Various human races (see RACE) have developed genetic defenses 
against MALARIA.  African and Mediterranean peoples, for 
instance, have genes for altered hemoglobin, on which the 
parasite cannot thrive as easily as it does in normal 
hemoglobin.  People having one normal hemoglobin gene and one 
altered gene lessen their chances of acquiring MALARIA. 
Unfortunately, people who inherit two altered genes are 
subject to either SICKLE-CELL DISEASE or thalassemia--
particularly COOLEY'S ANEMIA--depending on the type of gene 
inherited. 

Cure and Prevention

No feasible methods of treatment for MALARIA were available 
before the 1630s, when Spanish missionaries discovered an 
extract from the bark of the South American cinchona tree. 
This extract, the alkaloid QUININE, and a related drug, 
QUINIDINE, were the only antimalarial drugs until the first 
half of the 20th century.  Current treatment includes the drug 
chloroquine as the first choice, as well as pyrimethamine and 
chloroguanide.  Parasites have developed resistance to 
chloroquine, however, although combined use of another drug, 
desipramine, is helping to overcome this resistance.  Fears of 
similar development of resistance have been expressed 
concerning the worldwide distribution in the later 1980s of 
another drug, halofantrine, and may lead to limitations being 
placed on a further new drug, mefloquine.  Researchers are 
exploring the usefulness of yet another drug, arteether, 
derived from a chemical called qinghaosu that was isolated by 
Chinese chemists from the herb Artemisia annua.  This herb has 
been used for centuries in traditional Chinese medicine to 
treat MALARIA and fever. 

The need for a MALARIA vaccine is urgent.  In 1984, using 
genetic engineering techniques, researchers made large 
quantities of MALARIA antigen available for testing.  At about 
the same time, a method was developed for diagnosing MALARIA 
in developing countries on a large scale, using a genetic 
probe for P. falciparum.  By the end of the 1980s, however, 
tests of various genetically engineered vaccines had not yet 
led to a practical vaccine for regular use. 

People who have had MALARIA often suffer relapses, which may 
occur years after the initial infection.  One theory 
concerning the origin of relapses is that they are due to 
persistence of parasites in the liver;  another theory is that 
relapses are caused by parasitized red blood cells that 
circulate in the blood for prolonged periods of time. 
 
\section*{MENINGITIS}
 MENINGITIS is a potentially fatal inflammation of the meninges,
or membranes, covering the brain and spinal cord.  The 
organisms, usually bacterial or viral, gain access to the 
cerebrospinal fluid and follow the space around vessels. 
MENINGITIS may result from head injuries and infections 
involving the eyes, ears, and nose.  It can be a complication 
of systemic disorders such as pneumonia and syphilis, both of 
which reach the brain via the bloodstream. 

The initial symptoms are extreme headache, aggravated or 
precipitated by turning the head or touching the chin to the 
chest, rapidly rising fever, stiffness of the neck, and 
irritability and drowsiness.  Further progression depends on 
the causative agent and the health of the host.  The patient 
may experience deafness, muscle weakness in the face, and 
other signs of nerve paralysis.  Convulsions, mental 
retardation, and behavioral disturbances may also occur and 
may remain in some cases.  Many patients recover fully. 

Diagnosis is often made by lumbar puncture (spinal tap), 
whereby direct access is gained to the site of infection. 
Special stains and culture of the extracted fluid will often 
identify the specific organism so that proper therapy may 
begin.  Patients are treated with antibiotics, and a vaccine 
against haemophilus influenzae type b was licensed for use in 
1985. 


\section*{PERTUSSIS}

Whooping cough, or PERTUSSIS, is an acute infectious disease 
that is a leading cause of death in infants unprotected by 
immunization. The infective agent, the bacterium Bordetella 
PERTUSSIS, is transmitted by inhalation of material expelled 
by the coughing of infected patients. After an incubation 
period of about a week, symptoms at first resemble those of 
the common cold; after 7 to 10 days, coughing increases and 
becomes distinctive. A series of short coughs of increasing 
intensity is followed by a long indrawing of breath with the 
characteristic crowing sound or "whoop." The patient's face 
often becomes blue during the frightening spasms; clear sticky 
mucus is expelled, and vomiting commonly occurs. Pneumonia is 
a dangerous complication. The younger the patient, the greater 
the risk of serious illness; most deaths from whooping cough 
occur in the first 6 months of life. Treatment includes 
careful nursing, sedatives, and plenty of fluids. Antibiotics 
are needed only for the treatment of complications, such as 
pneumonia. Prevention is by immunization with a vaccine 
containing killed PERTUSSIS bacteria, often given in 
combination with diphtheria and tetanus
vaccines.

\section*{AMEBIASIS} 

AMEBIASIS is a general name for human infections caused by the 
amoeba Entamoeba histolytica.  Intestinal infections alone are 
called amoebic DYSENTERY.  AMEBIASIS occurs worldwide--nearly 
3,000 cases were reported in the United States in a recent 
year--but is most prevalent in tropical regions.  In some 
areas more than half of the population is likely to develop 
AMEBIASIS at some time.  The parasite is usually ingested in 
encysted form in contaminated food or water.  Dysentery 
symptoms may take a week to a year or more to develop and may 
recur after long remissions;  some people remain symptomless 
but act as carriers.  Untreated AMEBIASIS can lead to stomach 
ulcers and peritonitis, and the amoeba may be carried in the 
blood to the liver and cause amebic hepatitis or ulcers.  
Abscesses may also develop in other organs or, rarely, the 
brain.  Several drugs are effective against AMEBIASIS, but 
liver and other complications may require surgery. 

\newpage
\section*{SHIGELLOSIS} 

Dysentery is an inflammation of the intestine that causes 
painful DIARRHEA and stools containing blood and mucus.  One 
type, bacillary dysentery, is caused by SHIGELLA bacteria;  
and another, amoebic dysentery, by a protozoan, Entamoeba 
histolytica.  Bacillary dysentery occurs mainly in children; 
the symptoms are a fever, abdominal cramps, and diarrhea that 
lasts about a week.  It severely affects infants, the elderly, 
and the malnourished, especially in the tropics, and may be 
fatal.  Generally, treatment consists of drinking large 
quantities of fluids, but antibiotics are given in severe 
cases.  The disease is transmitted by the stools of infected 
persons and is prevented by good hygiene.  Amoebic dysentery 
is an ulcerative condition of the intestine, from which the 
amoebas may invade the liver or, rarely, the skin and lungs, 
and cause abscesses.  Treatment is emetine injections or 
metronidazole taken orally. 


SHIGELLA Named for K. Shiga (1870-1957), a Japanese bacteriologist, 
SHIGELLA is a genus of rod-shaped bacteria.  The normal 
habitat of all species is the intestinal tract of humans, 
primates, and, rarely, dogs.  SHIGELLA can cause human 
DYSENTERY, a disease transmitted by infected food or water. 


\section*{SALMONELLOSIS} 
SALMONELLA is a genus of rod-shaped bacteria.  More than 1,500 
species have been identified, usually isolated from the 
intestinal tract of humans and animals.  SALMONELLA organisms 
are responsible for human gastrointestinal tract diseases (see 
GASTROINTESTINAL TRACT DISEASE) known as salmonelloses, which 
range from fairly mild attacks of fever, diarrhea, vomiting, 
and abdominal cramps to life-threatening illness if the 
bacteria cause excessive dehydration or spread to other organs.
Outbreaks may occur from eating meat from animals with strains 
of SALMONELLA made resistant by excessive use of ANTIBIOTICS 
in the feed.  SALMONELLA infections caused by improper food 
handling or unsanitary conditions include FOOD POISONING and 
TYPHOID FEVER.  The bacteria have been found often in chicken 
and eggs since the late 1980s. 


\end{document}
