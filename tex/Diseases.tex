\documentstyle{slides}

\begin{document}
{\LARGE \bf MALARIA}

Malaria is a widespread and often fatal protozoal disease.  It
occurs mostly in tropical regions of the world, such as Africa,
Central America, and Southeast Asia.  Experts estimate there are
more than 270 million new Malaria infections worldwide each year and
an estimated 2 million deaths.

In 1880, Charles Laveran observed a 
protozoal parasite in the blood of an afflicted patient. In 
1898, Ronald Ross found that the bite of the female mosquito
of the genus Anopheles transmits the parasites into the 
victim's bloodstream. 

\newpage 
{\LARGE \bf MENINGITIS}
  
 Meningitis is a potentially fatal inflammation of the meninges,
or membranes, covering the brain and spinal cord.  The 
organisms, usually bacterial or viral, gain access to the 
cerebrospinal fluid and follow the space around vessels. 

The initial symptoms are extreme headache, aggravated or 
precipitated by turning the head or touching the chin to the 
chest, rapidly rising fever, stiffness of the neck, and 
irritability and drowsiness.  Further progression depends on 
the causative agent and the health of the host.

\newpage
{\LARGE \bf PERTUSSIS}
  
Whooping cough, or Pertussis, is an acute infectious disease 
that is a leading cause of death in infants unprotected by 
immunization. The infective agent, the bacterium Bordetella 
Pertussis, is transmitted by inhalation of material expelled 
by the coughing of infected patients. After an incubation 
period of about a week, symptoms at first resemble those of 
the common cold; after 7 to 10 days, coughing increases and 
becomes distinctive. A series of short coughs of increasing 
intensity is followed by a long indrawing of breath with the 
characteristic crowing sound or "whoop."

\newpage
{\LARGE \bf AMEBIASIS} 

  
Amebiasis is a general name for human infections caused by the 
amoeba Entamoeba histolytica.  Intestinal infections alone are 
called amoebic dysentery.  Amebiasis occurs worldwide--nearly 
3,000 cases were reported in the United States in a recent 
year--but is most prevalent in tropical regions.  In some 
areas more than half of the population is likely to develop 
Amebiasis at some time.  The parasite is usually ingested in 
encysted form in contaminated food or water.  Dysentery 
symptoms may take a week to a year or more to develop and may 
recur after long remissions;  some people remain symptomless 
but act as carriers.

\newpage
{\LARGE \bf SHIGELLOSIS}


Shigella Named for K. Shiga (1870-1957), a Japanese bacteriologist,
Shigella is a genus of rod-shaped bacteria.  The normal habitat of
all species is the intestinal tract of humans, primates, and,
rarely, dogs.

Dysentery is an inflammation of the intestine that causes 
painful diarrhea and stools containing blood and mucus.  One 
type, bacillary dysentery, is caused by Shigella bacteria;  
and another, amoebic dysentery, by a protozoan, Entamoeba 
histolytica.  Bacillary dysentery occurs mainly in children; 
the symptoms are a fever, abdominal cramps, and diarrhea that 
lasts about a week.  It severely affects infants, the elderly, 
and the malnourished, especially in the tropics, and may be 
fatal.  

\newpage
{\LARGE \bf SALMONELLOSIS}

Salmonella organisms are responsible for human gastrointestinal
tract diseases known as salmonelloses, which range from fairly mild
attacks of fever, diarrhea, vomiting, and abdominal cramps to
life-threatening illness if the bacteria cause excessive dehydration
or spread to other organs.  Outbreaks may occur from eating meat
from animals with strains of Salmonella made resistant by excessive
use of antibiotics in the feed.  Salmonella infections caused by
improper food handling or unsanitary conditions include food
poisoning and Typhoid fever.  The bacteria have been found often in
chicken and eggs since the late 1980s.


\end{document}
