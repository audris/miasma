\documentstyle[12pt,times]{article} \title{Spread of 19 notifiable
diseases in The United States in 1962-1992} \author{Audris Mockus}
\date{\today} \maketitle
\begin{document}

\section{Introduction}

\section{Description of the data}

The data was obtained from NNDSS (National Notifiable Disease
Surveillance System). The dataset contains numbers of cases per
state per month for the period from January, 1962 until December,
1992.  There are 19 diseases in this dataset (for comparison, there
are approximately 60?  notifiable diseases).


\subsection{Description of the more common diseases}

SALMONELLOSIS with 975427 cases is the most popular among the
diseases is followed by SHIGELLOSIS (565468 cases), PERTUSSIS
(139162 cases), AMEBIASIS (120789 cases), MENING.INF. (76275 cases).
At the bottom of the list are Rabies with 30 cases, TYPHUS MURINE
with 1401 cases, POLIO, PARALYTIC with 1823 cases, LEPTOSPIROSIS
with 2166 cases.

SALMONELLOSIS is caused by SALMONELLA organisms. It is human
gastrointestinal tract disease. It ranges from fairly mild attack of
fever, diarrhea, vomiting, and abdominal cramps, to life threatening
illness if bacteria cause excessive dehydration or spread to other
organs. Outbreaks may occur from eating meat from animals with
strains of SALMONELLA made resistant by excessive use of antibiotics
in the feed. The bacteria have been found often in chicken and eggs
since the late 1980s.

SHIGELLOSIS is a form dysentery caused by SHIGELLa bacteria (it is a
genus of rod-shaped bacteria named after K. Shiga (1870-1957) a
Japanese bactiorologist. Bacillary dysentery occurs mainly in
children; the symptoms are a fever, abdominal cramps, and diarrhea
that lasts about a week.

PERTUSSIS or whooping cough is an acute infectious disease that is a
leading cause of death in infants unprotected by immunization. The
infective agent, the bacterium Bordetella pertussis, is transmitted
by inhalation of material expelled by the coughing of infected
patients.

AMEBIASIS is a general name for human infections caused by the
amoeba Entamoeba hystolytica.



\subsection{Inspection of the data}

\begin{table}
\begin{tabular}{|c|c|r|c|}\hline
  Disease Name & Code & Total Counts & Reporting Proportion \\ 
  \hline BRUCELLOSIS & 10020 & 6518 & 0.18047649167193 \\ DIPHTHERIA
  & 10040 & 4250 & 0.06936538056082 \\ MALARIA & 10130 & 34125 &
  0.40538688593717 \\ MENING. INF.  & 10150 & 76275 &
  0.72190596668775 \\ PERTUSSIS & 10190 & 139162 & 0.60489141893316
  \\ TETANUS & 10210 & 4822 & 0.14779675310984 \\ TULAREMIA & 10230
  & 6416 & 0.16540164452877 \\ TYPHOID FEVER & 10240 & 14549 &
  0.30033733923677 \\ ROCKY MT. SPOT. FEV. & 10250 & 20586 &
  0.24525616698292 \\ TYPHUS MURINE & 10260 & 1401 &
  0.03478810879190 \\ TRICHINOSIS & 10270 & 3497 & 0.08053974277883
  \\ HANSEN DISEASE & 10380 & 5933 & 0.09514020662028 \\ 
  LEPTOSPIROSIS & 10390 & 2166 & 0.07179000632511 \\ POLIO PARALYTIC
  & 10410 & 1823 & 0.04464473961627 \\ PSITTACOSIS & 10450 & 2966 &
  0.09261016234450 \\ RABIES HUMAN & 10460 & 30 & 0.00158127767235
  \\ SALMONELLOSIS & 11000 & 975427 & 0.96131140628294 \\ 
  SHIGELLOSIS & 11010 & 565468 & 0.89574109213577 \\ AMEBIASIS &
  11040 & 120789 & 0.47137887413029 \\ \hline
\end{tabular}
\protect\caption{The 19 diseases with their code names, total
  reported counts over the period 1962-1992, and reporting
  proportion which is the number of state-month-year combinations
  when the counts were reported out of total $51 \times (1992-1962 +
  1) \times 12$ cells, i.e., 51 states, 31 years, and 12 months}
\end{table}


The worst incidence rates (higher than $10^{-4}$ or more than 1 per
10,000 people per month) for the above mentioned diseases and time
period are given in the Table~\ref{worst}. Notice the how many times
Washington DC occurs in the table.

{\footnotesize
\begin{table}
  \begin{tabular}{|c|c|c|r|l|c|r|r|}
    \multicolumn{2}{c}{}& \multicolumn{3}{c}{Highest Incidence} &
    \multicolumn{3}{c}{Lowest Incidence} \\ \hline State & Disease &
    Year & Month & Incidence & Year & Month & Incidence \\ \hline
    Vermont & PERTUSSIS & 63 & 1 & 0.0001006347 & 92 & 11 &
    1.723769e-06\\ Idaho & PERTUSSIS & 88 & 3 & 0.0001080529 & 92 &
    11 & 9.69681e-07 \\ Montana & PERTUSSIS & 63 & 3 & 0.0001083051
    & 92 & 9 & 1.23735e-06 \\ Colorado & PERTUSSIS & 63 & 5 &
    0.0001145665 & 92 & 5 & 2.936257e-07\\ North Dakota & PERTUSSIS
    & 63 & 9 & 0.000135046 & 80 & 3 & 1.532553e-06\\ West Virginia &
    PERTUSSIS & 62 & 6 & 0.0001763787 & 79 & 12 & 5.13323e-07 \\ 
    Alaska & PERTUSSIS & 62 & 3 & 0.0002016088 & 92 & 6 &
    1.699088e-06\\ Delaware & PERTUSSIS & 86 & 4 & 0.0002854273 & 92
    & 12 & 1.449013e-06\\ Kansas & PERTUSSIS & 86 & 10 &
    0.0004117225 & 92 & 12 & 3.966213e-07\\\hline Virginia &
    SALMONELLOSIS & 73 & 12 & 0.0001036889 & 62 & 3 & 1.454979e-06\\ 
    Montana & SALMONELLOSIS & 80 & 7 & 0.0001053917 & 73 & 12 & 0 \\ 
    Vermont & SALMONELLOSIS & 69 & 8 & 0.0001076496 & 89 & 2 &
    1.785526e-06\\ Indiana & SALMONELLOSIS & 85 & 4 & 0.0001112306 &
    64 & 1 & 2.045508e-07\\ Delaware & SALMONELLOSIS & 91 & 9 &
    0.000118953 & 82 & 1 & 1.639562e-06\\ Maine & SALMONELLOSIS & 82
    & 11 & 0.0001194372 & 81 & 1 & 8.806569e-07\\ Kansas &
    SALMONELLOSIS & 62 & 9 & 0.0001221687 & 91 & 3 & 4.000194e-07\\ 
    Colorado & SALMONELLOSIS & 76 & 8 & 0.0001250644 & 71 & 8 &
    8.577812e-07\\ South Carolina & SALMONELLOSIS & 70 & 6 &
    0.0001284951 & 88 & 5 & 2.905887e-07\\ Connecticut &
    SALMONELLOSIS & 87 & 2 & 0.0001334236 & 62 & 2 & 1.135237e-06\\ 
    Hawaii & SALMONELLOSIS & 67 & 1 & 0.0001406808 & 90 & 3 &
    8.94632e-07 \\ Alaska & SALMONELLOSIS & 83 & 12 & 0.0001652875 &
    86 & 8 & 1.999216e-06\\ Illinois & SALMONELLOSIS & 85 & 3 &
    0.000234284 & 64 & 1 & 1.425527e-06\\ Oregon & SALMONELLOSIS &
    84 & 10 & 0.0002750513 & 78 & 5 & 1.178464e-06\\ Kentucky &
    SALMONELLOSIS & 67 & 2 & 0.0002837664 & 75 & 5 & 2.883683e-07\\ 
    North Dakota & SALMONELLOSIS & 70 & 8 & 0.000324362 & 83 & 12 &
    1.542598e-06\\ Nebraska & SALMONELLOSIS & 67 & 9 & 0.0006804923
    & 76 & 4 & 6.482726e-07\\\hline Montana & SHIGELLOSIS & 74 & 3 &
    0.000105829 & 92 & 10 & 1.237134e-06\\ Arizona & SHIGELLOSIS &
    68 & 4 & 0.0001127645 & 68 & 6 & 5.845372e-07\\ Connecticut &
    SHIGELLOSIS & 82 & 8 & 0.0001255497 & 77 & 7 & 3.232736e-07\\ 
    Oregon & SHIGELLOSIS & 65 & 8 & 0.0001433889 & 71 & 2 &
    4.613942e-07\\ New Mexico & SHIGELLOSIS & 90 & 12 & 0.0002327978
    & 62 & 5 & 2.064682e-06\\ Alaska & SHIGELLOSIS & 80 & 9 &
    0.0002654794 & 92 & 12 & 1.677503e-06\\ Vermont & SHIGELLOSIS &
    68 & 12 & 0.0003823573 & 92 & 12 & 1.722445e-06\\ Hawaii &
    SHIGELLOSIS & 70 & 8 & 0.001044749 & 91 & 11 &
    8.750254e-07\\\hline
\end{tabular}
\protect\caption{The worst monthly incidence rates (above one case
  per 10,000 population)}\label{worst}
\end{table}                                                                               
}


The database contains 112845 records of the form

{\scriptsize \vspace{5 pt}
\begin{tabular}{|c|c|c|c|c|c|}
  \hline $({\rm Year} - 1900)$ & Month & State Fips Code & Disease
  Code & Disease Name and Code & Number of Cases \\ \hline 62 & 1 &
  6 & 10020& BRUCELLOSIS 10020 & 9 \\\hline
\end{tabular}
\vspace{5 pt} } The month number 13 was used to identify that the
month was unknown.


Time plots of the diseases for California ant the United States are
in Figures~\ref{timePlotsCal} and~\ref{timePlotsUS}.  The six most
common diseases were then plotted for six differnt (in terms of
climat and/or population size) states.  Notice the gross reporting
problems for salmonellosis in Florida.

\begin{figure}
  \centering \setlength{\unitlength}{1 in}
  \begin{picture}(6,6)
    \put(-1,-2){\special{psfile=Calif.19.ps hscale=90 vscale=90}}
  \end{picture}
  \setlength{\unitlength}{1 pt} \protect \caption{The time behavior
    of the 19 diseases in California}\label{timePlotsCal}
\end{figure}

\begin{figure}
  \centering \setlength{\unitlength}{1 in}
  \begin{picture}(6,6)
    \put(-1,-2){\special{psfile=US.19.ps hscale=90 vscale=90}}
  \end{picture}
  \setlength{\unitlength}{1 pt} \protect \caption{The time behavior
    of the 19 diseases in the US}\label{timePlotsUS}
\end{figure}


\begin{figure}
  \centering \setlength{\unitlength}{1 in}
  \begin{picture}(6,6)
    \put(-1,-2){\special{psfile=Malaria.6.ps hscale=90 vscale=90}}
  \end{picture}
  \setlength{\unitlength}{1 pt} \protect \caption{The time behavior
    of malaria in six states}
\end{figure}

\begin{figure}
  \centering \setlength{\unitlength}{1 in}
  \begin{picture}(6,6)
    \put(-1,-2){\special{psfile=Mening.6.ps hscale=90 vscale=90}}
  \end{picture}
  \setlength{\unitlength}{1 pt} \protect \caption{The time behavior
    of meningitus in six states}
\end{figure}

\begin{figure}
  \centering \setlength{\unitlength}{1 in}
  \begin{picture}(6,6)
    \put(-1,-2){\special{psfile=Pert.6.ps hscale=90 vscale=90}}
  \end{picture}
  \setlength{\unitlength}{1 pt} \protect \caption{The time behavior
    of pertussis in six states}
\end{figure}

\begin{figure}
  \centering \setlength{\unitlength}{1 in}
  \begin{picture}(6,6)
    \put(-1,-2){\special{psfile=Salmon.6.ps hscale=90 vscale=90}}
  \end{picture}
  \setlength{\unitlength}{1 pt} \protect \caption{The time behavior
    of salmonellosis in six states}
\end{figure}

\begin{figure}
  \centering \setlength{\unitlength}{1 in}
  \begin{picture}(6,6)
    \put(-1,-2){\special{psfile=Shig.6.ps hscale=90 vscale=90}}
  \end{picture}
  \setlength{\unitlength}{1 pt} \protect \caption{The time behavior
    of shigellosis in six states}
\end{figure}

\begin{figure}
  \centering \setlength{\unitlength}{1 in}
  \begin{picture}(6,6)
    \put(-1,-2){\special{psfile=Ameb.6.ps hscale=90 vscale=90}}
  \end{picture}
  \setlength{\unitlength}{1 pt} \protect \caption{The time behavior
    of amebiasis in six states}
\end{figure}
\end{document}

